\documentclass[14pt, oneside, a4paper]{extreport}
\usepackage{extsizes}

\usepackage[T1, T2A]{fontenc}
\usepackage[utf8]{inputenc}
\usepackage[english,russian]{babel}
\usepackage{tempora}

\usepackage[left=30mm,right=10mm,top=20mm,bottom=20mm]{geometry}

\usepackage{setspace}
\onehalfspacing

\usepackage{indentfirst}
\setlength{\parindent}{1.25cm}

\usepackage{enumitem}
\setlist{nolistsep}

\renewcommand{\labelitemi}{---}
\renewcommand{\labelenumi}{\arabic{enumi})}

\usepackage{graphicx}
\graphicspath{ {./img/} }

\usepackage{listings}
\lstset{
	basicstyle=\footnotesize\ttfamily,
	numbers=left,
	numberstyle=\tiny,
	numbersep=5pt,
	tabsize=4,
	breaklines=true,
	frame=b,
	stringstyle=\ttfamily\ttfamily,
	showspaces=false,
	showtabs=false,
	xleftmargin=17pt,
	framexleftmargin=17pt,
	framexrightmargin=5pt,
	framexbottommargin=4pt,
	showstringspaces=false,
	inputencoding=utf8x,
	keepspaces=true,
	numberbychapter=false
}

\usepackage{amsmath}
\usepackage{amsfonts}
\usepackage{mathtools}

\usepackage{tikz}
\usepackage{pgfplots}

\usepackage[normalem]{ulem}

\usepackage{titlesec}

\titleformat{\chapter}[block]{\bfseries\normalsize\filcenter}{\thechapter}{1em}{}[
	]
\titlespacing\chapter{\parindent}{-2em}{1em}

\titleformat{\section}[hang]{\bfseries\normalsize}{\thesection}{1em}{}
\titlespacing\section{\parindent}{\parskip}{\parskip}

\titleformat{\subsection}[hang]{\bfseries\normalsize}{\thesubsection}{1em}{}
\titlespacing\subsection{\parindent}{\parskip}{\parskip}

\usepackage{slashbox}

\usepackage{caption}
\captionsetup[figure]{justification=centering}
\DeclareCaptionLabelSeparator{emdash}{\ ---\ }
\captionsetup[figure]{name={Рисунок},labelsep=emdash}
\captionsetup[lstlisting]{justification=raggedright, labelsep=emdash}
\captionsetup[table]{singlelinecheck=false, labelsep=emdash}

\counterwithout{figure}{chapter}
\counterwithout{table}{chapter}
\counterwithout{equation}{chapter}

\newenvironment{sequations} {
\begin{subequations}
\renewcommand{\theequation}{\arabic{parentequation}.\arabic{equation}}
}{
\end{subequations}
}

\usepackage{array}
\newenvironment{signstabular}[1][1]{
	\renewcommand*{\arraystretch}{#1}
	\tabular
}{
	\endtabular
}

\makeatletter
\g@addto@macro\@floatboxreset\centering
\makeatother

\makeatletter
\renewcommand\@biblabel[1]{#1.}
\makeatother

\usepackage{rotating}

\addto\captionsrussian{\renewcommand{\bibname}{Список использованных источников}}

\usepackage{diagbox}

\usepgfplotslibrary{external} 
\tikzexternalize

\usepackage{longtable}
\usepackage{pdfpages}
\usepackage{etoolbox}

\newcounter{totfigures}
\newcounter{tottables}
\newcounter{totpages}
\newcounter{totsources}

\providecommand\totfig{} 
\providecommand\tottab{} 
\providecommand\totpg{}
\providecommand\totsrc{}

\makeatletter
\AtEndDocument{
	\addtocounter{totfigures}{\value{figure}}
	\addtocounter{tottables}{\value{table}}
	\addtocounter{totpages}{\value{page}}
	\addtocounter{totsources}{\value{enumiv}}
	\immediate\write\@mainaux{
	  \string\gdef\string\totfig{\number\value{totfigures}}
	  \string\gdef\string\tottab{\number\value{tottables}}
	  \string\gdef\string\totpg{\number\value{totpages}}
	  \string\gdef\string\totsrc{\number\value{totsources}}
	}
}
\makeatother

\addto\captionsrussian{% Replace "english" with the language you use
  \renewcommand{\contentsname}%
    {Содержание}%
}


\begin{document}

\include{titlepage}
\setcounter{page}{2}

\chapter{Введение}

\textbf{Цель} лабораторной работы --- разработать программу шифровальной машины <<Энигма>>.
Составлен ряд \textbf{задач}.
\begin{enumerate}
	\item Ознакомиться с устройством работы <<Энигмы>>.
	\item На основе <<Энигмы>> разработать алгоритм, позволяющий шифровать бинарные файлы.
	\item Разработать программу, реализующую данный алгоритм.
	\item Провести тестирование разработанной программы.
\end{enumerate}

\chapter{Теория}

<<Энигма>> состоит из трёх основных компонентов: набора роторов, рефлектора и коммутационной панели, а также клавиатуры. При нажатии клавиши посылается сигнал, который проходит через коммутационную панель, потом через все роторы, отражается в рефлекторе и возвращается аналогичным путём, подсвечивая одну из лампочек-букв.

Коммутационная панель представляет собой набор гнёзд под каждую букву алфавита. В комплекте с <<Энигмой>> идёт набор проводов, позволяющий соединить любую пару таких <<слотов>> между собой, таким образом произведя замену этих символов друг на друга.

Роторы представляют собой диски, с наборами контактов с обеих сторон. Контакты одной стороны произвольным образом связаны с контактами другой. После нажатия клавиши на клавиатуре, первый ротор он совершает вращение на одну позицию. Когда любой из роторов совершает полный оборот, следующий ротор совершает одно вращение.

Рефлектор попарно соединяет все буквы алфавита, и, получив сигнал, <<разворачивает>> его, таким образов подсвечивая пользователю зашифрованную букву.

\chapter{Разработка программы}
Для реализации программы был выбран язык программирования C++.
Разработанная программа поддерживает три режима работы:
\begin{enumerate}
	\item работа с текстовыми файлами в кодировке Windows1251, содержащими текст только на русском языке;
	\item работа с текстовыми файлами в кодировке Windows1251, содержащими текст только на английском языке;
	\item работа с бинарными файлами.
\end{enumerate}


Запуск программы осуществляется через командную строку, в аргументах необходимо указать имена входного и выходного файлов.
Пример: ./enigma-console.exe input output

\section{Тестирование}

Тестирование разработанной программы проводилось методологией <<чёрного ящика>>.
В таблице~\ref{table:test} приведены примеры тестов.

\begin{table}[h]
	\begin{center}
		\captionsetup{justification=raggedright,singlelinecheck=off,margin=5mm}
		\caption{Тестирование}
		\begin{tabular}{| c | c | c |}
			\hline
			Входной файл & Результат  & Режим работы\\
			\hline
			пустой файл & пустой файл & все \\
			\hline
			Hello world & MGDZT URFGJ & eng \\
			\hline
			MGDZT URFGJ & HELLO WORLD & eng \\
			\hline
			Привет мирЪ & СХАЛЛУ ЫЗСУ & ru \\
			\hline
			СХАЛЛУ ЫЗСУ & ПРИВЕТ МИРЪ & ru \\
			\hline
			hello, world! Привет, мирЪ & <что-то на бинарном> & bin \\
			\hline
			<что-то на бинарном> & hello, world! Привет, мирЪ & bin \\
			\hline
			<zip архив> & <bin> & bin \\
			\hline
			<bin> & <zip архив> & bin\\
			\hline
		\end{tabular}
		\label{table:test}
	\end{center}
\end{table}


\end{document}
